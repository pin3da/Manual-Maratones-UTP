\documentclass[10pt,letterpaper,twocolumn,twosided]{article}

\usepackage[utf8]{inputenc}
\usepackage[spanish]{babel}
\usepackage{listings}
\usepackage[usenames,dvipsnames]{color}
\usepackage{amsmath}
\usepackage{verbatim}
\usepackage{hyperref}
\usepackage{color}
\usepackage{geometry}

\geometry{verbose,landscape,letterpaper,tmargin=2cm,bmargin=2cm,lmargin=1cm,rmargin=1cm}

\usepackage{listings}
\usepackage{color}

\definecolor{dkgreen}{rgb}{0,0.6,0}
\definecolor{gray}{rgb}{0.5,0.5,0.5}
\definecolor{mauve}{rgb}{0.58,0,0.82}

\lstset{frame=tb,
  language=C++,
  aboveskip=3mm,
  belowskip=3mm,
  showstringspaces=false,
  columns=flexible,
  basicstyle={\small\ttfamily},
  numbers=none,
  numberstyle=\tiny\color{gray},
  keywordstyle=\color{blue},
  commentstyle=\color{dkgreen},
  stringstyle=\color{mauve},
  breaklines=true,
  breakatwhitespace=true
  tabsize=2
}

\newcommand{\codigofuente}[1]{
    
    \verbatiminput{#1}
    
\dotfill
}

\setlength{\columnsep}{0.5in}
\setlength{\columnseprule}{1px}

\begin{document}

\title{Team reference - UTP - 0x7DD}
\author{Universidad Tecnológica de Pereira}
\maketitle
\tableofcontents
\lstloadlanguages{C++,Java}

%===============================================%
\section{Data structures}

\subsection{Fenwick Tree}

\begin{lstlisting}
public static class FenwickTree {
    long[] fenwickTree; int tam;
    public FenwickTree(int t) {
        fenwickTree = new long[t]; tam = t;
    }
    long query(int a, int b) {
        if (a == 0) {
            long sum = 0;
            for (; b >= 0; b = (b & (b + 1)) - 1)
              sum += fenwickTree[b];
            return sum;
        } 
        else { return (query(0, b) - query(0, a - 1));}
    }
    void increase(int k, long inc) 
    { for (; k < tam; k |= k + 1) fenwickTree[k] += inc;}
    void increaseRange(int a, int b, long val)
    {increase(a, val); increase(b + 1, -val);}
}
\end{lstlisting}

\subsection{SegmentTree RMQ}

\begin{lstlisting}
static class SegmentTree {
    long[] M;
    public SegmentTree(int size) {
        M = new long[size * 4 + 4];
        Arrays.fill(M, Long.MAX_VALUE);
    }
    //it's initially called update(1, 0, size - 1, pos, value)
    void update(int node, int b, int e, int pos, long value) {
        //if the current interval doesn't intersect 
        //the updated position return -1
        if (pos > e || pos < b)
            return;
        //if the current interval is the updated position
        //then update it
        if (b == e)
        {   M[node] = value; return; }
        update(2 * node, b, (b + e) / 2, pos, value);
        update(2 * node + 1, (b + e) / 2 + 1, e, pos, value);
        //update current value after updating childs
        M[node] = Math.min(M[2 * node], M[2 * node + 1]);
    }
    //it's initially called query(1, 0, size - 1, i, j)
    long query(int node, int b, int e, int i, int j) {
        long p1, p2;
        //if the current interval doesn't intersect 
        //the query interval return Long.MAX_VALUE
        if (i > e || j < b)
            return Long.MAX_VALUE;
        //if the current interval is completely included in 
        //the query interval return the value of this node
        if (b >= i && e <= j)
            return M[node];
        //compute the value from
        //left and right part of the interval
        p1 = query(2 * node, b, (b + e) / 2, i, j);
        p2 = query(2 * node + 1, (b + e) / 2 + 1, e, i, j);
        //join them to generate result
        long tmp = Math.min(p1, p2);
        return tmp;
    }
}
\end{lstlisting}

\subsection{RMQ Sparse table}

\begin{lstlisting}
// Preprocess
for (int i = 0; i < n; ++i) M[i][0] = i;
for (int j = 1, p = 2, q = 1; p <= n; ++j, p <<= 1, q <<= 1)
  for (int i = 0; i + p - 1 < n; ++i) {
    long long a = M[i][j - 1], b = M[i + q][j - 1];
    M[i][j] = nums[a] <= nums[b] ? a : b;
  }
  
// query in interval [b, e] (inclusive)
int k = log2(e - b + 1);
long long a = M[b][k], a2 = M[e + 1 - (1<<k)][k];
int idx = nums[a] <= nums[a2] ? a : a2;

\end{lstlisting}

\subsection{Splay Tree + lazy propagation}

\begin{lstlisting}

struct node{
  node *left, *right, *parent;
  int cur, vset, size, c[27]; // current character, value to set, size, ans for i-th character at c[i].
  int rev, set; // Values for propagation, reverse and set operations.
  node (int k) : cur(k), left(0), right(0), parent(0) , rev(0), set(0) {
    c[k] = 1;
    size = 1;
  }
  void set_val(int a) {
    cur = vset = a;
    set = 1;
    memset(c, 0, sizeof c);
    c[a] = size;
  }
  void reverse() {
    rev ^= 1;
    swap(left, right);
  }
  void update() {
    size = 1;
    memset(c, 0, sizeof c);
    c[cur] = 1;
    if (left) {
      size += left->size;
      for (int i = 0; i < 26; ++i) c[i] += left->c[i];
    }
    if (right) {
      size += right->size;
      for (int i = 0; i < 26; ++i) c[i] += right->c[i];
    }
  }
  void propagate() {
    if (rev) {
      rev = 0;
      if (left) left->reverse();
      if (right) right->reverse();
    }
    if (set) {
      set = 0;
      if (left) left->set_val(vset);
      if (right) right->set_val(vset);
    }
  }
};
struct splay_tree{
  node *root;
  void right_rot(node *x) {
    node *p = x->parent;
    if (x->parent = p->parent) {
      if (x->parent->left == p) x->parent->left = x;
      if (x->parent->right == p) x->parent->right = x;
    }
    if (p->left = x->right) p->left->parent = p;
    x->right = p;
    p->parent = x;
    p->update();
  }
  void left_rot(node *x) {
    node *p = x->parent;
    if (x->parent = p->parent) {
      if (x->parent->left == p) x->parent->left = x;
      if (x->parent->right == p) x->parent->right = x;
    }
    if (p->right = x->left) p->right->parent = p;
    x->left = p;
    p->parent = x;
    p->update();
  }
  void splay(node *x, node *fa = 0) {
    while( x->parent != fa and x->parent != 0) {
      node *p = x->parent;
      if (p->parent == fa)
        if (p->right == x)
          left_rot(x);
        else
          right_rot(x);
      else {
        node *gp = p->parent;
        if (gp->left == p)
          if (p->left == x)
            right_rot(x),right_rot(x);
          else
            left_rot(x),right_rot(x);
        else
          if (p->left == x)
            right_rot(x), left_rot(x);
          else
            left_rot(x), left_rot(x);
      }
    }
    x->update();
    if (fa == 0) root = x;
    else fa->update();
  }
  splay_tree(){ root = 0;};
  node *build(int a, int b) {
    if (b < a) return 0;
    int mid = (a + b)>>1;
    node *x = new node(line[mid] - 'a');
    x->left = build(a, mid - 1);
    if (x->left) x->left->parent = x;
    x->right = build(mid + 1, b);
    if (x->right) x->right->parent = x;
    x->update();
    return x;
  }
  node *find(int k) {
    node *cur = root;
    while (true) {
      cur->propagate();
      if (cur->left) {
        if (cur->left->size >= k) {
          cur = cur->left;
          continue;
        }
        k -= cur->left->size;
      }
      if ( k == 1) break;
      k--;
      cur = cur->right;
    }
    return cur;
  }
  void set_val(int a, int b, int c) {
    node *end = find(b + 1);
    node *begin = find(a - 1);
    splay(end); splay(begin, end);
    begin->right->set_val(c);
    begin->update();
    end->update();
  }
  void reverse(int a, int b) {
    node *end = find(b + 1);
    node *begin = find(a - 1);
    splay(end); splay(begin, end);
    begin->right->reverse();
  }
  int query(int a, int b, int pos) {
    node *end = find(b + 1);
    node *begin = find(a - 1);
    splay(end); splay(begin, end);
    return begin->right->c[pos];
  }
  void good_bye(node *x) {
    if (x == 0) return;
    good_bye(x->left);
    good_bye(x->right);
    delete (x);
    x = 0;
  }
  void print(node *a) {
    if (!a) return;
    print(a->left);
    cout<<(char)(a->cur + 'a')<<" - ";
    print(a->right);
  }
};
\end{lstlisting}

\section{Graphs}

\subsection{LCA}

\begin{lstlisting}
// T[i] : Parent of node i in the tree
void process3(int N, int T[MAXN], int P[MAXN][LOGMAXN]) {
  int i, j;
  for (i = 0; i < N; i++)
      for (j = 0; 1 << j < N; j++)
          P[i][j] = -1;
//the first ancestor of every node i is T[i]
  for (i = 0; i < N; i++)
      P[i][0] = T[i];
//bottom up dynamic programing
  for (j = 1; 1 << j < N; j++)
     for (i = 0; i < N; i++)
         if (P[i][j - 1] != -1)
             P[i][j] = P[P[i][j - 1]][j - 1];
}
// L[i] : level of node i (dist to root)
int query(int N, int P[MAXN][LOGMAXN], int T[MAXN], int L[MAXN], int p, int q) {
  int tmp, log, i;
  //if p is situated on a higher level than q then we swap them
  if (L[p] < L[q])
      tmp = p, p = q, q = tmp;
  //we compute the value of [log(L[p)]
  for (log = 1; 1 << log <= L[p]; log++);
  log--;
  //we find the ancestor of node p situated on the same level
  //with q using the values in P
  for (i = log; i >= 0; i--)
      if (L[p] - (1 << i) >= L[q])
          p = P[p][i];
  if (p == q)
      return p;
  //we compute LCA(p, q) using the values in P
  for (i = log; i >= 0; i--)
      if (P[p][i] != -1 && P[p][i] != P[q][i])
          p = P[p][i], q = P[q][i];
  return T[p];
}
\end{lstlisting}

\subsection{Heavy Light Decomposition}

\begin{lstlisting}

// Heavy-Light Decomposition
struct TreeDecomposition {
  vector<int> g[MAXN], c[MAXN];
  int s[MAXN]; // subtree size
  int p[MAXN]; // parent id
  int r[MAXN]; // chain root id
  int t[MAXN]; // index used in segtree/bit/...
  int d[MAXN]; // depht
  int ts;      // time stamp
  void dfs(int v, int f) {
    p[v] = f;
    s[v] = 1;
    if (f != -1) d[v] = d[f] + 1;
    else d[v] = 0;
    for (int i = 0; i < g[v].size(); ++i) {
      int w = g[v][i];
      if (w != f) {
        dfs(w, v);
        s[v] += s[w];
      }
    }
  }
  void hld(int v, int f, int k) {
    t[v] = ts++;
    c[k].push_back(v);
    r[v] = k;
    int x = 0, y = -1;
    for (int i = 0; i < g[v].size(); ++i) {
      int w = g[v][i];
      if (w != f) {
        if (s[w] > x) {
          x = s[w];
          y = w;
        }
      }
    }
    if (y != -1) {
      hld(y, v, k);
    }
    for (int i = 0; i < g[v].size(); ++i) {
      int w = g[v][i];
      if (w != f && w != y) {
        hld(w, v, w);
      }
    }
  }
  void init(int n) {
    for (int i = 0; i < n; ++i) {
      g[i].clear();
    }
  }
  void add(int a, int b) {
    g[a].push_back(b);
    g[b].push_back(a);
  }
  void build() {
    ts = 0;
    dfs(0, -1);
    hld(0, 0, 0);
  }
};
\end{lstlisting}


\subsection{Articulation Points}

\begin{lstlisting}
typedef string node;
typedef map<node, vector<node> > graph;
typedef char color;
const color WHITE = 0, GRAY = 1, BLACK = 2;
graph g;
map<node, color> colors;
map<node, int> d, low;
set<node> cameras; //articulation points
int timeCount;
// Uso: Para cada nodo u:
// colors[u] = WHITE, g[u] = Aristas salientes de u.
// Funciona para grafos no dirigidos.
void dfs(node v, bool isRoot = true){
  colors[v] = GRAY;
  d[v] = low[v] = ++timeCount;
  const vector<node> &neighbors = g[v];
  int count = 0;
  for (int i=0; i<neighbors.size(); ++i){
    if (colors[neighbors[i]] == WHITE){
      //(v, neighbors[i]) is a tree edge
      dfs(neighbors[i], false);
      if (!isRoot && low[neighbors[i]] >= d[v]){
        //current node is an articulation point
        cameras.insert(v);
      }
      low[v] = min(low[v], low[neighbors[i]]);
      ++count;
    }else{ // (v, neighbors[i]) is a back edge
      low[v] = min(low[v], d[neighbors[i]]);
    }
  }
  if (isRoot && count > 1){
    //Is root and has two neighbors in the DFS-tree
    cameras.insert(v);
  }
  colors[v] = BLACK;
}
\end{lstlisting}


\subsection{Bridges}
\begin{lstlisting}
int visited[MP];
int prev[MP], low[MP], d[MP];
vector< vector<int> > g;
vector< pair<int,int> > bridges;
int n, ticks;
void dfs(int u){
  visited[u] = true;
  d[u] = low[u] = ticks++;
  for (int i=0; i<g[u].size(); ++i){
    int v = g[u][i];
    if (prev[u] != v){
      if(!visited[v]){
         prev[v] = u;
         dfs(v);
         if (d[u] < low[v]){
           bridges.push_back(make_pair(min(u,v),max(u,v)));
         }
         low[u] = min(low[u], low[v]);
      }else{
         low[u] = min(low[u], d[v]);
      }
    }
  }
}
/**Example of use **/
memset(visited,false,sizeof(visited));
memset(prev,-1,sizeof(prev));
g.assign(n, vector<int>());
bridges.clear();
if (n == 0){ printf("0 critical links\n"); continue; }
for (int i=0; i<n; ++i){
  int node, deg;
  scanf("%d (%d)", &node, &deg);
  g[node].resize(deg);
  for (int k=0, x; k<deg; ++k){
    scanf("%d", &x);
    g[node][k] = x;
  }
}
ticks = 0;
for (int i=0; i<n; ++i){
  if (!visited[i]){
    dfs(i);
  }
}
sort(bridges.begin(), bridges.end());
printf("%d critical links\n", bridges.size());
foreach(p, bridges)
  printf("%d - %d\n", p->first, p->second);
\end{lstlisting}

\subsection{Stable Marriage}

\begin{lstlisting}
int N,pref_men[MAX_N][MAX_N],pref_women[MAX_N][MAX_N];
int inv[MAX_N][MAX_N],cont[MAX_N],wife[MAX_N],husband[MAX_N];
void stable_marriage(){
    for(int i = 0 ; i < N ; i++)
        for(int j = 0;j < N; j++)
            inv[i][pref_women[i][j]] = j;
   
    fill(cont,cont+N,0);
    fill(wife,wife+N,-1);
    fill(husband,husband+N,-1);
    queue<int> Q;
    for(int i = 0; i < N; i++) Q.push(i);
    int m,w;
    while(!Q.empty()){
        m = Q.front();
        w = pref_men[m][cont[m]];
        if(husband[w] == -1){
            wife[m] = w;
            husband[w] = m;
            Q.pop();
        }else{
            if( inv[w][m] < inv[w][husband[w]] ){
                wife[m] = w;
                wife[husband[w]] = -1;
                Q.pop();
                Q.push(husband[w]);
                husband[w] = m;
            }
        }
        cont[m]++;
    }
}
\end{lstlisting}


\subsection{SCC - Tarjan}

\begin{lstlisting}
vector<int> g[MAXN];
int d[MAXN], low[MAXN], scc[MAXN];
bool stacked[MAXN];
stack<int> s;
int ticks, current_scc;
void tarjan(int u){
  d[u] = low[u] = ticks++;
  s.push(u);
  stacked[u] = true;
  const vector<int> &out = g[u];
  for (int k=0, m=out.size(); k<m; ++k){
    const int &v = out[k];
    if (d[v] == -1){
      tarjan(v);
      low[u] = min(low[u], low[v]);
    }else if (stacked[v]){
      low[u] = min(low[u], low[v]);
    }
  }
  if (d[u] == low[u]){
    int v;
    do{
      v = s.top();
      s.pop();
      stacked[v] = false;
      scc[v] = current_scc;
    }while (u != v);
    current_scc++;
  }
}
\end{lstlisting}


\subsection{Dinnic}

\begin{lstlisting}
// Adjacency list implementation of Dinic's blocking flow algorithm.
// This is very fast in practice, and only loses to push-relabel flow.
// Running time:
//     O(|V|^2 |E|)
//
// INPUT: 
//     - graph, constructed using AddEdge()
//     - source
//     - sink
// OUTPUT:
//     - maximum flow value
//     - To obtain the actual flow values, look at all edges with
//       capacity > 0 (zero capacity edges are residual edges).
const int INF = 2000000000;

struct Edge {
  int from, to, cap, flow, index;
  Edge(int from, int to, int cap, int flow, int index) :
    from(from), to(to), cap(cap), flow(flow), index(index) {}
};

struct Dinic {
  int N;
  vector<vector<Edge> > G;
  vector<Edge *> dad;
  vector<int> Q;
  
  Dinic(int N) : N(N), G(N), dad(N), Q(N) {}
  
  void AddEdge(int from, int to, int cap) {
    G[from].push_back(Edge(from, to, cap, 0, G[to].size()));
    if (from == to) G[from].back().index++;
    G[to].push_back(Edge(to, from, 0, 0, G[from].size() - 1));
  }

  long long BlockingFlow(int s, int t) {
    fill(dad.begin(), dad.end(), (Edge *) NULL);
    dad[s] = &G[0][0] - 1;
    
    int head = 0, tail = 0;
    Q[tail++] = s;
    while (head < tail) {
      int x = Q[head++];
      for (int i = 0; i < G[x].size(); i++) {
	Edge &e = G[x][i];
	if (!dad[e.to] && e.cap - e.flow > 0) {
	  dad[e.to] = &G[x][i];
	  Q[tail++] = e.to;
	}
      }
    }
    if (!dad[t]) return 0;

    long long totflow = 0;
    for (int i = 0; i < G[t].size(); i++) {
      Edge *start = &G[G[t][i].to][G[t][i].index];
      int amt = INF;
      for (Edge *e = start; amt && e != dad[s]; e = dad[e->from]) {
	if (!e) { amt = 0; break; }
	amt = min(amt, e->cap - e->flow);
      }
      if (amt == 0) continue;
      for (Edge *e = start; amt && e != dad[s]; e = dad[e->from]) {
	e->flow += amt;
	G[e->to][e->index].flow -= amt;
      }
      totflow += amt;
    }
    return totflow;
  }

  long long GetMaxFlow(int s, int t) {
    long long totflow = 0;
    while (long long flow = BlockingFlow(s, t))
      totflow += flow;
    return totflow;
  }
};

\end{lstlisting}

\subsection{Max Flow - Push Relabel}

\begin{lstlisting}
static class Edge {
    int from, to, index;
    long cap, flow;

    Edge(int fromi, int toi, long capi, long flowi, int indexi) {
        from = fromi; to = toi; cap = capi; flow = flowi; 
        index = indexi;
    }
}

static class PushRelabel {
    int N; int ans; boolean[] active;
    ArrayList <Edge> [] G; long[] excess; int[] dist, count;
    ArrayDeque <Integer> Q = new ArrayDeque <Integer> ();

    PushRelabel(int N1) {
        N = N1; G = new ArrayList[N]; active = new boolean[N];
        for(int i = 0; i < N; i++)
            G[i] = new ArrayList <Edge> ();
        excess = new long[N]; dist = new int[N]; 
        count = new int[2 * N];
    }
    
    void AddEdge(int from, int to, int cap) {
        int cambio = from == to ? 1 : 0;
        G[from].add(new Edge(from, to, cap, 0, G[to].size() + cambio));
        G[to].add(new Edge(to, from, 0, 0, G[from].size() - 1));
    }

    void Enqueue(int v) { 
        if (!active[v] && excess[v] > 0){active[v] = true; Q.add(v);} 
    }

    void Push(Edge e) {
        long amt = Math.min(excess[e.from], e.cap - e.flow);
        if(dist[e.from] <= dist[e.to] || amt == 0) return;
        e.flow += amt; G[e.to].get(e.index).flow -= amt;
        excess[e.to] += amt;  excess[e.from] -= amt; Enqueue(e.to);
    }

    void Gap(int k) {
        for(int v = 0; v < N; v++)  {
            if(dist[v] < k)  continue;
            count[dist[v]]--; dist[v] = Math.max(dist[v], N + 1);
            count[dist[v]]++; Enqueue(v);
        }
    }

    void Relabel(int v) {
        count[dist[v]]--; dist[v] = 2 * N;
        for (Edge e : G[v]) 
            if (e.cap - e.flow > 0) 
                dist[v] = Math.min(dist[v], dist[e.to] + 1);
        count[dist[v]]++; Enqueue(v);
    }

    void Discharge(int v) {
        for(Edge e : G[v]) { if(excess[v] <= 0) break; Push(e);}
        if(excess[v] > 0) 
            {if(count[dist[v]] == 1) Gap(dist[v]); else Relabel(v);}
    }

    long GetMaxFlow(int s, int t) {
        count[0] = N - 1; count[N] = 1; dist[s] = N; 
        active[s] = active[t] = true;
        for (Edge e : G[s]) { excess[s] += e.cap; Push(e); }
        while (!Q.isEmpty()) { int v = Q.poll(); active[v] = false; 
                               Discharge(v);}
        long totflow = 0;
        for (Edge e : G[s]) totflow += e.flow; return totflow;
    }
}
\end{lstlisting}


\subsection{Mincost matching}

\begin{lstlisting}
double MinCostMatching(const VVD &cost, VI &Lmate, VI &Rmate) {
  int n = int(cost.size());

  // construct dual feasible solution
  VD u(n);
  VD v(n);
  for (int i = 0; i < n; i++) {
    u[i] = cost[i][0];
    for (int j = 1; j < n; j++) u[i] = min(u[i], cost[i][j]);
  }
  for (int j = 0; j < n; j++) {
    v[j] = cost[0][j] - u[0];
    for (int i = 1; i < n; i++) v[j] = min(v[j], cost[i][j] - u[i]);
  }
  
  // construct primal solution satisfying complementary slackness
  Lmate = VI(n, -1);
  Rmate = VI(n, -1);
  int mated = 0;
  for (int i = 0; i < n; i++) {
    for (int j = 0; j < n; j++) {
      if (Rmate[j] != -1) continue;
      if (fabs(cost[i][j] - u[i] - v[j]) < 1e-10) {
	Lmate[i] = j;
	Rmate[j] = i;
	mated++;
	break;
      }
    }
  }
  
  VD dist(n);
  VI dad(n);
  VI seen(n);
  
  // repeat until primal solution is feasible
  while (mated < n) {
    
    // find an unmatched left node
    int s = 0;
    while (Lmate[s] != -1) s++;
    
    // initialize Dijkstra
    fill(dad.begin(), dad.end(), -1);
    fill(seen.begin(), seen.end(), 0);
    for (int k = 0; k < n; k++) 
      dist[k] = cost[s][k] - u[s] - v[k];
    
    int j = 0;
    while (true) {
      
      // find closest
      j = -1;
      for (int k = 0; k < n; k++) {
	if (seen[k]) continue;
	if (j == -1 || dist[k] < dist[j]) j = k;
      }
      seen[j] = 1;
      
      // termination condition
      if (Rmate[j] == -1) break;
      
      // relax neighbors
      const int i = Rmate[j];
      for (int k = 0; k < n; k++) {
	if (seen[k]) continue;
	const double new_dist = dist[j] + cost[i][k] - u[i] - v[k];
	if (dist[k] > new_dist) {
	  dist[k] = new_dist;
	  dad[k] = j;
	}
      }
    }
    
    // update dual variables
    for (int k = 0; k < n; k++) {
      if (k == j || !seen[k]) continue;
      const int i = Rmate[k];
      v[k] += dist[k] - dist[j];
      u[i] -= dist[k] - dist[j];
    }
    u[s] += dist[j];
    
    // augment along path
    while (dad[j] >= 0) {
      const int d = dad[j];
      Rmate[j] = Rmate[d];
      Lmate[Rmate[j]] = j;
      j = d;
    }
    Rmate[j] = s;
    Lmate[s] = j;
    
    mated++;
  }
  
  double value = 0;
  for (int i = 0; i < n; i++)
    value += cost[i][Lmate[i]];
  
  return value;
}
\end{lstlisting}

\subsection{Min cost - Max flow}

\begin{lstlisting}
const L INF = numeric_limits<L>::max() / 4;

struct MinCostMaxFlow {
  int N;
  VVL cap, flow, cost;
  VI found;
  VL dist, pi, width;
  VPII dad;

  MinCostMaxFlow(int N) : 
    N(N), cap(N, VL(N)), flow(N, VL(N)), cost(N, VL(N)), 
    found(N), dist(N), pi(N), width(N), dad(N) {}
  
  void AddEdge(int from, int to, L cap, L cost) {
    this->cap[from][to] = cap;
    this->cost[from][to] = cost;
  }
  
  void Relax(int s, int k, L cap, L cost, int dir) {
    L val = dist[s] + pi[s] - pi[k] + cost;
    if (cap && val < dist[k]) {
      dist[k] = val;
      dad[k] = make_pair(s, dir);
      width[k] = min(cap, width[s]);
    }
  }

  L Dijkstra(int s, int t) {
    fill(found.begin(), found.end(), false);
    fill(dist.begin(), dist.end(), INF);
    fill(width.begin(), width.end(), 0);
    dist[s] = 0;
    width[s] = INF;
    
    while (s != -1) {
      int best = -1;
      found[s] = true;
      for (int k = 0; k < N; k++) {
        if (found[k]) continue;
        Relax(s, k, cap[s][k] - flow[s][k], cost[s][k], 1);
        Relax(s, k, flow[k][s], -cost[k][s], -1);
        if (best == -1 || dist[k] < dist[best]) best = k;
      }
      s = best;
    }

    for (int k = 0; k < N; k++)
      pi[k] = min(pi[k] + dist[k], INF);
    return width[t];
  }

  pair<L, L> GetMaxFlow(int s, int t) {
    L totflow = 0, totcost = 0;
    while (L amt = Dijkstra(s, t)) {
      totflow += amt;
      for (int x = t; x != s; x = dad[x].first) {
        if (dad[x].second == 1) {
          flow[dad[x].first][x] += amt;
          totcost += amt * cost[dad[x].first][x];
        } else {
          flow[x][dad[x].first] -= amt;
          totcost -= amt * cost[x][dad[x].first];
        }
      }
    }
    return make_pair(totflow, totcost);
  }
};
\end{lstlisting}


\section{Math}

\subsection{Chinese remainder theorem}

\begin{lstlisting}
void extended_euclid(ll a, ll b, ll &x, ll &y, ll &g) {
    x = 0; y = 1; g = b;
    ll m, n, q, r;
    for (ll u=1, v=0; a != 0; g=a, a=r) {
        q = g / a; r = g % a;
        m = x-u*q; n = y-v*q;
        x=u; y=v; u=m; v=n;
    }
}
/*****************************
* Find z such that            *
* z % x[i] = a[i] for all i.  *
******************************/
ll chinese_remainder_theorem(vector<ll> ns, vector<ll> as){
    int k  = ns.size();
    ll N = 1, x = 0, r, s, g;
    for (int i = 0; i < k; ++i) N *= ns[i];
    for (int i = 0; i < k; ++i) {
        extended_euclid(ns[i], N/ns[i], r, s, g);
        x += as[i]*s*(N/ns[i]);
        x %= N;
    }
    if (x < 0) x += N;
    return x;
}
\end{lstlisting}

\subsection{utilities}

\begin{lstlisting}
// returns d = gcd(a,b); finds x,y such that d = ax + by
int extended_euclid(int a, int b, int &x, int &y) {  
  int xx = y = 0;
  int yy = x = 1;
  while (b) {
    int q = a/b;
    int t = b; b = a%b; a = t;
    t = xx; xx = x-q*xx; x = t;
    t = yy; yy = y-q*yy; y = t;
  }
  return a;
}
// finds all solutions to ax = b (mod n)
VI modular_linear_equation_solver(int a, int b, int n) {
  int x, y;
  VI solutions;
  int d = extended_euclid(a, n, x, y);
  if (!(b%d)) {
    x = mod (x*(b/d), n);
    for (int i = 0; i < d; i++)
      solutions.push_back(mod(x + i*(n/d), n));
  }
  return solutions;
}
// computes b such that ab = 1 (mod n), returns -1 on failure
int mod_inverse(int a, int n) {
  int x, y;
  int d = extended_euclid(a, n, x, y);
  if (d > 1) return -1;
  return mod(x,n);
}
// computes x and y such that ax + by = c; on failure, x = y =-1
void linear_diophantine(int a, int b, int c, int &x, int &y) {
  int d = gcd(a,b);
  if (c%d) {
    x = y = -1;
  } else {
    x = c/d * mod_inverse(a/d, b/d);
    y = (c-a*x)/b;
  }
}
\end{lstlisting}

\subsection{Euler Totient} 

\begin{lstlisting}
for (int i = 0; i < MP; ++i)phi[i] = i;
for (int i = 0; primes[i] <= 5000000; ++i) {
    phi[primes[i]] = primes[i] - 1;
    for (i64 j = 2 * primes[i]; j <= 5000000; j += primes[i]) {
        phi[j] = phi[j] * (primes[i]-1);
        phi[j] = phi[j] / primes[i];
    }
}
\end{lstlisting}

\section{Strings}

\subsection{Z-Algorithm}

\begin{lstlisting}
vector<int> compute_z(const string &s){
  int n = s.size();
  vector<int> z(n,0);
  int l,r;
  r = l = 0;
  for(int i = 1; i < n; ++i){
    if(i > r) {
      l = r = i;
      while(r < n and s[r - l] == s[r])r++;
      z[i] = r - l;r--;
    }else{
      int k = i-l;
      if(z[k] < r - i +1) z[i] = z[k];
      else {
        l = i;
        while(r < n and s[r - l] == s[r])r++;
        z[i] = r - l;r--;
      }
    }
  }
  return z;
}
\end{lstlisting}

\subsection{KMP}

\begin{lstlisting}
vector<int> compute_prefix_function(string p){
    vector<int> pi(p.size());
    pi[0]=-1;
    int k=-1;
    for(int i=1;i<p.size();i++){
        while(k>=0 && p[k+1]!=p[i]) k=pi[k];
        if (p[k+1]==p[i])k++;
        pi[i]=k;
    }
    return pi;
}

int KMP_Matcher(string p,string t){
    vector<int> pi=compute_prefix_function(p);
    int q=-1;
    int last = q;
    for(int i=0;i<t.size();i++){
        while(q>=0 && p[q+1]!=t[i]) q=pi[q];
        if (p[q+1]==t[i]) q++;
        last = q;
        if (q==p.size() - 1){
            q=pi[q];
        }
    }
    return last;
}
\end{lstlisting}


\subsection{Aho Corasik}

\begin{lstlisting}
// Max number of states in the matching machine.
// Should be equal to the sum of the length of all keywords.
const int MAXS = 6 * 50 + 10;

// Number of characters in the alphabet.
const int MAXC = 26;

// Output for each state, as a bitwise mask.
// Bit i in this mask is on if the keyword with index i
// appears when the machine enters this state.
int out[MAXS];

// Used internally in the algorithm.
int f[MAXS]; // Failure function
int g[MAXS][MAXC]; // Goto function, or -1 if fail.

// Builds the string matching machine.
//
// words - Vector of keywords. The index of each keyword is
// important:
// "out[state] & (1 << i)" is > 0 if we just found
// word[i] in the text.
// lowestChar - The lowest char in the alphabet.
// Defaults to 'a'.
// highestChar - The highest char in the alphabet.
// Defaults to 'z'.
// "highestChar - lowestChar" must be <= MAXC,
// otherwise we will access the g matrix outside
// its bounds and things will go wrong.
//
// Returns the number of states that the new machine has.
// States are numbered 0 up to the return value - 1, inclusive.
int buildMatchingMachine(const vector<string> &words,
                         char lowestChar = 'a',
                         char highestChar = 'z') {
    memset(out, 0, sizeof out);
    memset(f, -1, sizeof f);
    memset(g, -1, sizeof g);
    
    int states = 1; // Initially, we just have the 0 state
        
    for (int i = 0; i < words.size(); ++i) {
        const string &keyword = words[i];
        int currentState = 0;
        for (int j = 0; j < keyword.size(); ++j) {
            int c = keyword[j] - lowestChar;
            if (g[currentState][c] == -1) {
                // Allocate a new node
                g[currentState][c] = states++;
            }
            currentState = g[currentState][c];
        }
        // There's a match of keywords[i] at node currentState.
        out[currentState] |= (1 << i);
    }
    
    // State 0 should have an outgoing edge for all characters.
    for (int c = 0; c < MAXC; ++c) {
        if (g[0][c] == -1) {
            g[0][c] = 0;
        }
    }

    // Now, let's build the failure function
    queue<int> q;
    // Iterate over every possible input
    for (int c = 0; c <= highestChar - lowestChar; ++c) {
        // All nodes s of depth 1 have f[s] = 0
        if (g[0][c] != -1 and g[0][c] != 0) {
            f[g[0][c]] = 0;
            q.push(g[0][c]);
        }
    }
    while (q.size()) {
        int state = q.front();
        q.pop();
        for (int c = 0; c <= highestChar - lowestChar; ++c) {
            if (g[state][c] != -1) {
                int failure = f[state];
                while (g[failure][c] == -1) {
                    failure = f[failure];
                }
                failure = g[failure][c];
                f[g[state][c]] = failure;

                // Merge out values
                out[g[state][c]] |= out[failure];
                q.push(g[state][c]);
            }
        }
    }

    return states;
}

int findNextState(int currentState, char nextInput,
                                    char lowestChar = 'a') {
    int answer = currentState;
    int c = nextInput - lowestChar;
    while (g[answer][c] == -1) answer = f[answer];
    return g[answer][c];
}


// How to use this algorithm:
//
// buildMatchingMachine(keywords, 'a', 'z');
// int currentState = 0;
// for (int i = 0; i < text.size(); ++i) {
//   currentState = findNextState(currentState, text[i], 'a');
//
//   Nothing new, let's move on to the next character.
//   if (out[currentState] == 0) continue;
//
//   for (int j = 0; j < keywords.size(); ++j) {
//     if (out[currentState] & (1 << j)) {
//       //Matched keywords[j]
//       cout << "Keyword " << keywords[j]
//       << " appears from "
//       << i - keywords[j].size() + 1
//       << " to " << i << endl;
//     }
//   }
// }
//
// The output of this program is:
//
// Keyword his appears from 1 to 3
// Keyword he appears from 4 to 5
// Keyword she appears from 3 to 5
// Keyword hers appears from 4 to 7
\end{lstlisting}

\subsection{Suffix array}

\begin{lstlisting}
struct SuffixArray {
  const int L;
  string s;
  vector<vector<int> > P;
  vector<pair<pair<int,int>,int> > M;

  SuffixArray(const string &s) : L(s.length()), s(s), P(1, vector<int>(L, 0)), M(L) {
    for (int i = 0; i < L; i++) P[0][i] = int(s[i]);
    for (int skip = 1, level = 1; skip < L; skip *= 2, level++) {
      P.push_back(vector<int>(L, 0));
      for (int i = 0; i < L; i++) 
	M[i] = make_pair(make_pair(P[level-1][i], i + skip < L ? P[level-1][i + skip] : -1000), i);
      sort(M.begin(), M.end());
      for (int i = 0; i < L; i++) 
	P[level][M[i].second] = (i > 0 && M[i].first == M[i-1].first) ? P[level][M[i-1].second] : i;
    }    
  }

  vector<int> GetSuffixArray() { return P.back(); }

  // returns the length of the longest common prefix of s[i...L-1] and s[j...L-1]
  int LongestCommonPrefix(int i, int j) {
    int len = 0;
    if (i == j) return L - i;
    for (int k = P.size() - 1; k >= 0 && i < L && j < L; k--) {
      if (P[k][i] == P[k][j]) {
	i += 1 << k;
	j += 1 << k;
	len += 1 << k;
      }
    }
    return len;
  }
};
\end{lstlisting}


\section{Geometry}

\subsection{Picks theorem}

\[ Area = B/2 + I - 1 \]

$Area$: Area of a polygon, $B$: Lattice points in the boundary, $I$: Lattice points inside.


\subsection{Cross product}

\[ U \times W = (u_2 w_3 - u_3 w_2, u_3 w_1 - u_1 w_3, u_1 w_2 - u_2 w_1) \]

\subsection{Rotation}

CCW Rotation of $(x, r)$ around the origin.

\[ ( x_r, y_r ) = ( x * \cos \theta - y * \sin \theta, x * \sin \theta + y * \cos \theta ) \]

To rotate counterclockwise $v = (x,y,z)$ an angle $\theta$ around $k$ (unitary vector that represents the rotation axis) use:($Rodrigues rotation formula$).

\[ \mathbf{v}_\mathrm{rot} = \mathbf{v} \cos\theta + (\mathbf{k} \times \mathbf{v})\sin\theta
  + \mathbf{k} (\mathbf{k} \cdot \mathbf{v}) (1 - \cos\theta).\]
  
\subsection{Cartesian coordinates from latitude and longitude}

\[ ( x, y, z ) = ( r \cos \phi \cos \lambda, r \cos \phi \sin \lambda, r \sin \phi ) \]

\subsection{Triangles}

Heron's Formula:
\[ Area = \sqrt{s (s - a) (s - b) (s - c)}, s = 0.5 * perimeter \]
The radius $r$ of the triangle's \textbf{inner circle} with area $A$ and semiperimeter $s$ is $r = A/s$.
The radius $R$ of the triangle's \textbf{Outer circle} with area $A$ and sides $a$,$b$ and $c$ is $R = abc/4A$.
Law of cosines:
\[ c^2 = a^2 + b^2 - 2ab \cos \gamma\]
where $\gamma$ denotes the angle contained between sides of lengths $a$ and $b$ and opposite the side of length $c$.

\subsection{Volumes and Areas}

Sphere Volume: $\frac{4}{3} \pi r^3$. Sphere Surface area: $4 \pi r^2$.
Lateral Surface area of a right circular cone: $LSA = \pi  r  \sqrt{r^2 + h^2}$.
  
\subsection{Lines and Segments Intersection}


\begin{lstlisting}
point intersectionbtwlines(point p1,point p2,point p3,point p4){
    point p2_p1 = p2.sub(p1);
    point p4_p3 = p4.sub(p3);
    double den = p2_p1.cross(p4_p3);
    if (Math.abs(den)<eps) return null; //parallel or coincident 
    point p1_p3 = p1.sub(p3);
    double num = p4_p3.cross(p1_p3);
    double ua = num/den;
    double num2 = p2_p1.cross(p1_p3)
    double ub = num2/den;
    return p2_p1.multbyscalar(ua).add(p1);
}
\end{lstlisting}

This function computes the intersection of the line passing through p1 and p2 with the line passing through p3 and p4. If \verb+den+ is equal to zero then the lines are parallel. If \verb+den+, \verb+num+ and \verb+num2+ are equal to zero then the lines are coincident. If the \textbf{intersection of line segments} is required then it is only necessary to test if \verb+ua+ and \verb+ub+ lie between 0 and 1.

\subsection{Circle Cirle Intersection}

\begin{lstlisting}
ArrayList<Point> CircleCircleIntersection(Point a, Point b, 
double r, double R) {
    ArrayList<Point> ret = new ArrayList<Point>();
    double d = a.sub(b).norm();
    if ( (d > r+R+eps) || (d+Math.min(r, R) < Math.max(r, R)-eps)) 
        return ret;
    double x = (d*d-R*R+r*r)/(2*d);
    double y = Math.sqrt(Math.max(0.0,r*r-x*x));
    Point v = b.sub(a).multbyscalar(1 / d);
    Point xx = v.multbyscalar(x).add(a);
    Point yy = v.RotateCCW90().multbyscalar(y);
    ret.add(xx.add(yy));
    if (y > eps)
        ret.add(xx.sub(yy));
    return ret;
}
\end{lstlisting}

\subsection{Half Plane Intersection}

\begin{lstlisting}
point[] sutherland_hodgman(point[] line, point[] poly,int flag){
    ArrayList<point> output=new ArrayList<point>();
    point S=poly[poly.length-1];
    for(int i=0;i<poly.length;i++){
        point E=poly[i];
        double ecross=line[1].sub(line[0]).cross(E.sub(line[0]));
        double scross=line[1].sub(line[0]).cross(S.sub(line[0]));
        if (Math.abs(ecross)<eps)
            output.add(E);
        else if (ecross*flag>eps){
            if (scross*flag<-eps)
                output.add(intersectionbtwlines(line[0]
                    ,line[1],S,E));
            output.add(E);
        }
        else if(scross*flag>eps)
            output.add(intersectionbtwlines(line[0],line[1],S,E));
        S=E;
    }
    point[] ret=new point[output.size()];
    return output.toArray(ret);
}
\end{lstlisting}

This function computes the intersection of a half plane with a polygon. The result is given in the same order of \verb+poly+. The variable \verb+flag+ only can take 1.0 and -1.0 as values. If \verb+flag+ is equal to 1.0 then the halfplane considered is to the left of the directed vector \verb+line+.

\subsection{Polygon area}

\begin{lstlisting}
double polygonarea(point[] a, int n){
    double r=0;
    for(int i=0;i<n;i++)
        r+=a[i].cross(a[(i+1)%n]);
    return Math.abs(r/2.0);
}
\end{lstlisting}

\subsection{Polygon's Centroid}

\begin{lstlisting}
static Point centroid(Point[] a, int n){
    Point ret = new Point(0.0, 0.0);
    double scale = 6.0 * signedpolygonarea(a, n);
    for(int i = 0; i < n; i++){
        int j = (i + 1) % n;
        Point t = a[i].add(a[j]);
        t = t.multbyscalar(a[i].cross(a[j]));
        ret = ret.add(t);
    }
    return ret.multbyscalar(1.0 / scale);
}
\end{lstlisting}

This function computes the centroid of a non-intersecting closed polygon. The function \verb+signedpolygonarea+ is equal to the \verb+polygonarea+ function but it doesn't take the  absolute value of \verb+r+.

\subsection{Point in Polygon}

\begin{lstlisting}
boolean insideconvexpolygon(point[] p, int n, point t) {
    int mask=0;
    for(int i=0;i<n;i++){
        int z=ccw(p[i],p[(i+1)%n],t);
        if (z<0) mask |= 1;
        else if (z>0) mask |= 2;
        else if (z==0) return belongstoedge(p[i],p[(i+1)%n],t); 
        if (mask==3) return false;
    }
    return true;
}

static double cross(Point2D a, Point2D b) 
    {return (a.getX() * b.getY()) - (a.getY() * b.getX());}

static boolean contains(Point2D p) {
    int cnt = 0;
    for(Line2D line : lines) {
        Point2D curr = subtract(line.getP1(), p); 
        Point2D next = subtract(line.getP2(), p);
        if(curr.getY() > next.getY()) 
        { Point2D temp = curr; curr = next; next = temp; }
        if(curr.getY() < 0 && 0 <= next.getY() && 
        cross(next, curr) >= 0) cnt++;
        if(is_point_online(line.getP1(), line.getP2(), p))
            return true;
      }
	  return  cnt % 2 == 1;
}
\end{lstlisting}

\begin{lstlisting}
// determine if point is in a possibly non-convex polygon (by William
// Randolph Franklin); returns 1 for strictly interior points, 0 for
// strictly exterior points, and 0 or 1 for the remaining points.
// Note that it is possible to convert this into an *exact* test using
// integer arithmetic by taking care of the division appropriately
// (making sure to deal with signs properly) and then by writing exact
// tests for checking point on polygon boundary
bool PointInPolygon(const vector<PT> &p, PT q) {
  bool c = 0;
  for (int i = 0; i < p.size(); i++){
    int j = (i+1)%p.size();
    if ((p[i].y <= q.y && q.y < p[j].y || 
      p[j].y <= q.y && q.y < p[i].y) &&
      q.x < p[i].x + (p[j].x - p[i].x) * (q.y - p[i].y) / (p[j].y - p[i].y))
      c = !c;
  }
  return c;
}

// determine if point is on the boundary of a polygon
bool PointOnPolygon(const vector<PT> &p, PT q) {
  for (int i = 0; i < p.size(); i++)
    if (dist2(ProjectPointSegment(p[i], p[(i+1)%p.size()], q), q) < EPS)
      return true;
    return false;
}
\end{lstlisting}

\subsection{Min-distance between 3D segments}

\begin{lstlisting}
double distanciabtwsegments(Point[] r, Point[] s) {
    Point v = s[1].sub(s[0]);
    Point u = r[1].sub(r[0]);
    Point w = r[0].sub(s[0]);
    double a = u.dot(u), b = u.dot(v), c = v.dot(v), 
    d = u.dot(w), e = v.dot(w);
    double D = a*c - b*b;
    double sc, sN, sD = D;
    double tc, tN, tD = D;
    if (D < epsilon) {
        sN = 0; sD = 1; tN = e; tD = c;
    } 
    else {
        sN = (b*e - c*d);
        tN = (a*e - b*d);
        if (sN < 0) {
            sN = 0; tN = e; tD = c;
        } else if (sN > sD) {
            sN = sD; tN = e + b; tD = c;
        }
    }
    if (tN < 0) {
        tN = 0;
        if (-d < 0) {
            sN = 0;
        } else if (-d > a) {
            sN = sD;
        } else {
            sN = -d;
            sD = a;
        }
    } 
    else if (tN > tD) {
        tN = tD;
        if ((-d + b) < 0) {
            sN = 0;
        } else if (-d + b > a) {
            sN = sD;
        } else {
            sN = -d + b;
            sD = a;
        }
    }
    sc = (Math.abs(sN) < epsilon ? 0 : sN / sD);
    tc = (Math.abs(tN) < epsilon ? 0 : tN / tD);
    Point dP = w.add(u.multbyscalar(sc)).sub(v.multbyscalar(tc));
    return dP.norm();
}
\end{lstlisting}

\subsection{Min-distance between 3D lines}


\begin{lstlisting}
double distancebtwlines(Point[] r, Point[] s) {
    Point u = r[1].sub(r[0]); Point v = s[1].sub(s[0]); 
    Point w = s[0].sub(r[0]);
    double a = u.dot(u), b = u.dot(v), c = v.dot(v), 
    d = u.dot(w), e = v.dot(w);
    double D = a*c - b*b, sc, tc;
    if (D < epsilon) {
        sc = 0;
        tc = (b > c) ? d/b : e/c;
    } else {
        sc = (b*e - c*d) / D;
        tc = (a*e - b*d) / D;
    }
    Point dP = w.add(u.multbyscalar(sc)).sub(v.multbyscalar(tc));
    return dP.norm();
}
\end{lstlisting}

\subsection{Convex Hull}

\begin{lstlisting}

static class cmp2 implements Comparator<point> {
    @Override
    public int compare(point f, point s) {
        double tmp=ccw(first_point,f,s);
        if (Math.abs(tmp)<eps) {
            if (f.sub(first_point).norm() < s.sub(first_point).norm())
                return -1;
            else
                return 1;
        }
        return (tmp>0)?1:-1;
    }
}


static point[] convexhull(point[] in,int n) {
    ArrayList<point> hull=new ArrayList<point>(n);
    int index = 0;
    for(int i = 1; i < n; i++)
        if (in[i].y < in[index].y || 
        (Math.abs(in[i].y - in[index].y)<eps && in[i].x < in[index].x))
            index = i;
    swap(in, 0, index);
    first_point=in[0];
    Arrays.sort(in,1,n,new cmp2());
    if (n<=3) {
        for(int i=0; i<n; i++)
            hull.add(in[i]);
        point[] a=new point[hull.size()];
        return hull.toArray(a);
    }
    // this is for a redundant CH
    int rev = n - 1;
    for(int r = n - 2; r >= 1; r--)
        if (ccw(first_point,in[r],in[rev]) == 0)
            rev = r;
        else
            break;
    if (rev != 1)  reverse(in, rev, n);
    //
    hull.add(first_point);
    hull.add(in[1]);
    int top=1;
    int i=2;
    while(i<n) { // >= for a nonredundant CH
        if (top>0 && ccw(hull.get(top-1),hull.get(top),in[i]) > eps) {
            hull.remove(top);
            top=top-1;
        }
        else {
            top=top+1;
            hull.add(in[i]);
            i++;
        }
    }
    point[] a=new point[hull.size()];
    return hull.toArray(a);
}

static double ccw(point a,point b,point c) {
    return a.sub(b).cross(c.sub(b));
}
\end{lstlisting}

\subsection{Convex Hull Trick}

\begin{lstlisting}
static class Line implements Comparable<Line>{
    long m,b;
    public Line(long mm, long bb){
        m = mm; b = bb;
    }
    long eval(long x){
        return m * x + b;
    }
    Frac getIntersection(Line o){
        if (o == null) return minf;
        return new Frac(o.b - b, m - o.m);
    }
    public int compareTo(Line o) {
        return Long.valueOf(m).compareTo(o.m);
    }
}

static class Hull{
    TreeMap<Line, Frac> lines;
    TreeMap<Frac, Line> intervals;
    
    public Hull(){
        lines =new TreeMap<Line, Frac>();
        intervals = new TreeMap<Frac, Line>();
    }
    
    long query(long x){
        Frac tmp = intervals.floorKey(new Frac(x, 1));
        Line maximal = intervals.get(tmp);
        return maximal.eval(x);
    }
    
    boolean checkbad(Line l1, Line l2, Line l3){
        Frac f13 = new Frac(l3.b - l1.b, l1.m - l3.m);
        Frac f12 = new Frac(l1.b - l2.b, l2.m - l1.m);    
        return f13.compareTo(f12) <= 0;
    }
    
    void Add(long m, long b){
        Line l = new Line(m, b);
        Line floor = lines.floorKey(l);
        // is there a line in the hull with same slope?
        if (floor!= null && floor.m == m){
            // this is maximization problem so the higher the 
            // y-intercept the better.
            if (floor.b < b){
                intervals.remove(lines.get(floor));
                lines.remove(floor);
            }
            else
                return;
        }
        Line lower = lines.lowerKey(l);
        Line higher = lines.higherKey(l);
        // is this line important for the hull?
        if (lower != null && higher != null){
            if (checkbad(floor,l , higher))
                return;
        }
        // keep the invariant to the left
        while(true){
            Line l2 = lines.lowerKey(l);
            if (l2 == null) break;
            Line l1 = lines.lowerKey(l2);
            if (l1 == null) break;
            if (!checkbad(l1,l2,l)) break;
            intervals.remove(lines.get(l2));
            lines.remove(l2);
        }
        //keep the invariant to the right
        while(true){
            Line l4 = lines.higherKey(l);
            if (l4 == null) break;
            Line l5 = lines.higherKey(l4);
            if (l5 == null) break;
            if (!checkbad(l,l4,l5)) break;
            intervals.remove(lines.get(l4));
            lines.remove(l4);
        }
        Line forward = lines.higherKey(l);
        if (forward != null){
            Frac intersection = l.getIntersection(forward);
            intervals.remove(lines.get(forward));
            lines.put(forward, intersection);
            intervals.put(intersection, forward);
        }
        lower = lines.lowerKey(l);
        Frac intersection = l.getIntersection(lower);
        lines.put(l, intersection);
        intervals.put(intersection, l);
    }
}
\end{lstlisting}

\verb+Frac+ stands for a simple fraction class, the comparison function between fractions should be defined carefully. \verb+minf+ represents minus infinity. This code represents an upper envelope (for max-queries). To build a lower envelope the comparison function between lines must be inverted and the \verb+<+ must be changed by \verb+>+  in the \verb+Add+ function.

\subsection{Smallest enclosing disk}

\begin{lstlisting}
/*
 * Calculates the sed of a set of Points. Call initially with R = empty set.
 * P is the set of points in the plane. R is the set of points lying on the boundary of the current circle.
 */

function sed(P,R) 
{
    if (P is empty or |R| = 3) then
         D := calcDiskDirectly(R)
    else
        choose a p from P randomly;
        D := sed(P - {p}, R);
        if (p lies NOT inside D) then
            D := sed(P - {p}, R u {p});
    return D;
}
\end{lstlisting}

\subsection{Great circle distance}

\begin{lstlisting}
static double greatCircleDistance(double latitudeS, double longitudeS, double latitudeF, double longitudeF, double r) {
	latitudeS = Math.toRadians(latitudeS);
	latitudeF = Math.toRadians(latitudeF);
	longitudeS = Math.toRadians(longitudeS);
	longitudeF = Math.toRadians(longitudeF);
	double deltaLongitude = longitudeF - longitudeS;
	double a = Math.cos(latitudeF) * Math.sin(deltaLongitude);
	double b = Math.cos(latitudeS) * Math.sin(latitudeF);
	b -= Math.sin(latitudeS) * Math.cos(latitudeF) 
         * Math.cos(deltaLongitude);
	double c = Math.sin(latitudeS) * Math.sin(latitudeF);
	c += Math.cos(latitudeS) * Math.cos(latitudeF) 
         * Math.cos(deltaLongitude);
	return Math.atan(Math.sqrt(a * a + b * b) / c) * r;
}
\end{lstlisting}


\section{Dp Optimizations}

\subsection{Divide and conquer optimization}

\begin{lstlisting}
static void solve(int k, int a, int b, int L, int R){
	if (b < a)
		return;
	int mid = (a + b)/2;
	int best = Integer.MAX_VALUE;
	int bestidx = -1;
	for(int i = Math.max(mid, L); i <= R; i++){
		if (dp[i + 1][k - 1] == Integer.MAX_VALUE)
			continue;
		int value = dp[i + 1][k - 1] + some_function(mid, i);
		if (value < best){
			best = value;
			bestidx = i;
		}
	}
	dp[mid][k] = best;
	solve(k, a, mid - 1, L, bestidx);
	solve(k, mid + 1, b, bestidx, R);
}
\end{lstlisting}

\subsection{Convex hull trick}

\begin{lstlisting}
struct line{
    long long a,b;
    
    line(){}
    line(long long _a, long long _b):
        a(_a),b(_b){}
};

long long a[MAXM],suma[MAXM];
long long dp[2][MAXM];
line H[MAXM];
int sz,pos;

bool check(line &l1, line &l2, line &l3){
    return (l3.b - l2.b) * (l1.a - l3.a) >= (l3.b - l1.b) * (l2.a - l3.a);
}

void insert(long long a, long long b){
    line l(a,b);
    while(sz >= 2 && !check(H[sz - 2],H[sz - 1],l)) --sz;
    H[sz] = l;
    ++sz;
}

long long eval(int ind, long long x){
    return H[ind].a * x + H[ind].b;
}

long long query(long long x){
    while(pos + 1 < sz && eval(pos,x) > eval(pos + 1,x)) ++pos;
    return eval(pos,x);
}
\end{lstlisting}

\section{Misc}

\subsection{Longest Increasing subsecuence}

\begin{lstlisting}
typedef vector<int> VI;
typedef pair<int,int> PII;
typedef vector<PII> VPII;

#define STRICTLY_INCREASNG

VI LongestIncreasingSubsequence(VI v) {
  VPII best;
  VI dad(v.size(), -1);
  
  for (int i = 0; i < v.size(); i++) {
#ifdef STRICTLY_INCREASNG
    PII item = make_pair(v[i], 0);
    VPII::iterator it = lower_bound(best.begin(), best.end(), item);
    item.second = i;
#else
    PII item = make_pair(v[i], i);
    VPII::iterator it = upper_bound(best.begin(), best.end(), item);
#endif
    if (it == best.end()) {
      dad[i] = (best.size() == 0 ? -1 : best.back().second);
      best.push_back(item);
    } else {
      dad[i] = dad[it->second];
      *it = item;
    }
  }
  
  VI ret;
  for (int i = best.back().second; i >= 0; i = dad[i])
    ret.push_back(v[i]);
  reverse(ret.begin(), ret.end());
  return ret;
}
\end{lstlisting}

\subsection{FFT}

Call with $\theta = \frac{2\pi}{N}$, where $N$ is a power of 2. For inverse transform use
$\theta = -\frac{2\pi}{N}$ instead and divide every element by $N$. Tests:

\begin{itemize}
\item \verb+[0.25,0.25,0.25,0.25] = [1,0,0,0]+
\item \verb+[-1,0,1,0] = [0,-2,0,-2]+
\end{itemize}

\begin{lstlisting}
struct cpx {
  cpx(){}
  cpx(double aa):a(aa){}
  cpx(double aa, double bb):a(aa),b(bb){}
  double a;
  double b;
  double modsq(void) const {
    return a * a + b * b;
  }
  cpx bar(void) const {
    return cpx(a, -b);
  }
};
cpx operator +(cpx a, cpx b) {
  return cpx(a.a + b.a, a.b + b.b);
}
cpx operator *(cpx a, cpx b) {
  return cpx(a.a * b.a - a.b * b.b, a.a * b.b + a.b * b.a);
}
cpx operator /(cpx a, cpx b) {
  cpx r = a * b.bar();
  return cpx(r.a / b.modsq(), r.b / b.modsq());
}
cpx EXP(double theta) {
  return cpx(cos(theta),sin(theta));
}
void fft(int n, double theta, cpx* a) {
    cpx basew = EXP(theta);
    for (int m = n; m >= 2; m >>= 1) {
        int mh = m >> 1;
        cpx w = cpx(1,0);
        for (int i = 0; i < mh; i++) {    		
            for (int j = i; j < n; j += m) {
                int k = j + mh;
                cpx x = a[j] - a[k];
                a[j] = a[j] + a[k];
                a[k] = w * x;
            }
            w = w * basew;
        }
        theta *= 2; basew = basew*basew;
    }
    int i = 0;
    for (int j = 1; j < n - 1; j++) {
        for (int k = n >> 1; k > (i ^= k); k >>= 1);
        if (j < i) swap(a[i], a[j]);
    }
}
\end{lstlisting}

\subsection{Integer sequences}

\subsubsection{Sums}

\begin{tabular}{ll}
$\sum_{k=0}^{n}k^{2}=n(n+1)(2n+1)/6$   & \tabularnewline
$\sum_{k=0}^{n}k^{3}=n^{2}(n+1)^{2}/4$  & \tabularnewline
$\sum_{k=0}^{n}k^{4}=(6n^{5}+15n^{4}+10n^{3}-n)/30$   & \tabularnewline
$\sum_{k=0}^{n}k^{5}=(2n^{6}+6n^{5}+5n^{4}-n^{2})/12$  & \tabularnewline
$\sum_{k=0}^{n}x^{k}=(x^{n+1}-1)/(x-1)$  &    \tabularnewline
$\sum_{k=0}^{n}kx^{k}=(x-(n+1)x^{n+1}+nx^{n+2})/(x-1)^{2}$  & \tabularnewline
\end{tabular}

\subsubsection{Bell Numbers}

Is the number of partitions of a set of $n$ elements ($S=\{1, 1, 2, 5, 15, 52, 203, 877, 4140, 21147, 115975, ...\}$).
They follow the recurrence $B_{n+1} = \sum_{k=0}^n{ {n \choose k}B_k } $.

The algorithm to get the sequence produces a triangle as:

\begin{verbatim}
 1
 1  2
 2  3  5
 5  7 10 15
15 20 27 37 52
\end{verbatim}

Where the first number of a row is the last of the previous one and each other number is
produced by adding the previous number in the row by the number above it.

\subsubsection{Catalan numbers}

The sequence C=\{1, 1, 2, 5, 14, 42, 132, 429, 1430, 4862, 16796, 58786, 208012, 742900, 
2674440, 9694845, 35357670, 129644790, 477638700, 1767263190\} is called Catalan sequence. They
obey the following formula:
\[C_n = {2n \choose n} - {2n \choose n+1}\]
And the following recurence relation:
\[ C_{n+1} = \sum_{i=0}^n{C_i C_{n-1}} \]

\subsection{Combinations and permutations}

\begin{tabular}[t]{|l |c |r|}
\hline
Is order important? & Is repetition allowed? & Formula \\
\hline
Yes & Yes & $ {n^r} $ \\
\hline
No & Yes & ${\frac{(n+r-1)!}{r!(n-1)!}} $ \\
\hline
Yes & No & $ {\frac{n!}{(n-r)!}} $ \\
\hline
No & No & $ {\frac{n!}{r!(n-r)!}} $  \\
\hline
\end{tabular}

\subsubsection{Others}

\paragraph{Derangements} A derangement is a permutation of the elements of a set such that none of the elements appear in their original position. Number of permutations of $n=0,1,2,\dots$ elements without fixed points. Sequence: $1,0,1,2,9,44,265,1854,14833,\dots$
Recurrence: $D_{n}=(n-1)(D_{n-1}+D_{n-2})=nD_{n-1}+(-1)^{n}$. Corollary:
number of permutations with exactly $k$ fixed points is ${n \choose k}D_{n-k}$. 


\paragraph{Double factorial} Permutations of the multiset $\left\{ 1,1,2,3,\dots n,n\right\} $
such that for each $k$, all the numbers between two occurrences of
$k$ in the permutation are greater than $k$. $(2n-1)!!=\prod_{k=1}^{n}\left(2k-1\right)$.


\paragraph{Multinomial theorem} $(a_{1}+\dots+a_{k})^{n}=\sum{n \choose n_{1},\dots,n_{k}}\ a_{1}^{n_{1}}\dots a_{k}^{n_{k}}$,
where $n_{i}\ge0$ and $\sum n_{i}=n$. \\
 ${n \choose n_{1},\dots,n_{k}}=M(n_{1},\dots,n_{k})=\frac{n!}{n_{1}!\dots n_{k}!}$.
$M(a,\dots,b,c,\dots)=M(a+\dots+b,c,\dots)M(a,\dots,b)$

\paragraph{Luccas theorem} Let $n,k,p\in naturals$ and $p$ be a prime number. Then
\[ \binom nk \equiv \prod_{j=0}^t \binom{n_j}{k_j} \mod p, \]
where $n_j$ and $k_j$ are the $j$-th digits of the numbers $n$ and $k$ in base $p$, respectively, and $t$ is the length of max($n$, $m$) in base $p$.
        
\paragraph{Pythagorean triples} Integer solutions of $x^{2}+y^{2}=z^{2}$
All relatively prime triples are given by: $x=2mn,y=m^{2}-n^{2},z=m^{2}+n^{2}$
where $m>n,\gcd(m,n)=1$ and $m\not\equiv n\pmod{2}$. All other triples
are multiples of these. Equation $x^{2}+y^{2}=2z^{2}$ is equivalent
to $(\frac{x+y}{2})^{2}+(\frac{x-y}{2})^{2}=z^{2}$.

\paragraph{Euler's phi function} $\varphi(n)=|\{m\in{N},m\le n,\gcd(m,n)=1\}|$.
$\quad\varphi(n)=n\cdot\prod_{p|n}\left(1-\frac{1}{p}\right)$\\
 $\varphi(mn)=\frac{\varphi(m)\varphi(n)\gcd(m,n)}{\varphi(\gcd(m,n))}$.
\quad{}$\varphi(p^{a})=p^{a-1}(p-1)$. \quad{}$\sum_{d|n}\varphi(d)=\sum_{d|n}\varphi(\frac{n}{d})=n$.

\paragraph{Euler's theorem} $a^{\varphi(n)}\equiv1\pmod{n}$, if $\gcd(a,n)=1$.

\subsection{Simplex}

\begin{lstlisting}
// Two-phase simplex algorithm for solving linear programs of the form
//
//     maximize     c^T x
//     subject to   Ax <= b
//                  x >= 0
//
// INPUT: A -- an m x n matrix
//        b -- an m-dimensional vector
//        c -- an n-dimensional vector
//        x -- a vector where the optimal solution will be stored
//
// OUTPUT: value of the optimal solution (infinity if unbounded
//         above, nan if infeasible)
//
// To use this code, create an LPSolver object with A, b, and c as
// arguments.  Then, call Solve(x).
struct LPSolver {
  int m, n;
  VI B, N;
  VVD D;

  LPSolver(const VVD &A, const VD &b, const VD &c) : 
    m(b.size()), n(c.size()), N(n+1), B(m), D(m+2, VD(n+2)) {
    for (int i = 0; i < m; i++) for (int j = 0; j < n; j++) D[i][j] = A[i][j];
    for (int i = 0; i < m; i++) { B[i] = n+i; D[i][n] = -1; D[i][n+1] = b[i]; }
    for (int j = 0; j < n; j++) { N[j] = j; D[m][j] = -c[j]; }
    N[n] = -1; D[m+1][n] = 1;
  }
	   
  void Pivot(int r, int s) {
    for (int i = 0; i < m+2; i++) if (i != r)
      for (int j = 0; j < n+2; j++) if (j != s)
	D[i][j] -= D[r][j] * D[i][s] / D[r][s];
    for (int j = 0; j < n+2; j++) if (j != s) D[r][j] /= D[r][s];
    for (int i = 0; i < m+2; i++) if (i != r) D[i][s] /= -D[r][s];
    D[r][s] = 1.0 / D[r][s];
    swap(B[r], N[s]);
  }

  bool Simplex(int phase) {
    int x = phase == 1 ? m+1 : m;
    while (true) {
      int s = -1;
      for (int j = 0; j <= n; j++) {
	if (phase == 2 && N[j] == -1) continue;
	if (s == -1 || D[x][j] < D[x][s] || D[x][j] == D[x][s] && N[j] < N[s]) s = j;
      }
      if (D[x][s] >= -EPS) return true;
      int r = -1;
      for (int i = 0; i < m; i++) {
	if (D[i][s] <= 0) continue;
	if (r == -1 || D[i][n+1] / D[i][s] < D[r][n+1] / D[r][s] ||
	    D[i][n+1] / D[i][s] == D[r][n+1] / D[r][s] && B[i] < B[r]) r = i;
      }
      if (r == -1) return false;
      Pivot(r, s);
    }
  }

  DOUBLE Solve(VD &x) {
    int r = 0;
    for (int i = 1; i < m; i++) if (D[i][n+1] < D[r][n+1]) r = i;
    if (D[r][n+1] <= -EPS) {
      Pivot(r, n);
      if (!Simplex(1) || D[m+1][n+1] < -EPS) return -numeric_limits<DOUBLE>::infinity();
      for (int i = 0; i < m; i++) if (B[i] == -1) {
	int s = -1;
	for (int j = 0; j <= n; j++) 
	  if (s == -1 || D[i][j] < D[i][s] || D[i][j] == D[i][s] && N[j] < N[s]) s = j;
	Pivot(i, s);
      }
    }
    if (!Simplex(2)) return numeric_limits<DOUBLE>::infinity();
    x = VD(n);
    for (int i = 0; i < m; i++) if (B[i] < n) x[B[i]] = D[i][n+1];
    return D[m][n+1];
  }
};
int main() {
  const int m = 4;
  const int n = 3;  
  DOUBLE _A[m][n] = {
    { 6, -1, 0 },
    { -1, -5, 0 },
    { 1, 5, 1 },
    { -1, -5, -1 }
  };
  DOUBLE _b[m] = { 10, -4, 5, -5 };
  DOUBLE _c[n] = { 1, -1, 0 };
  
  VVD A(m);
  VD b(_b, _b + m);
  VD c(_c, _c + n);
  for (int i = 0; i < m; i++) A[i] = VD(_A[i], _A[i] + n);
  LPSolver solver(A, b, c);
  VD x;
  DOUBLE value = solver.Solve(x);
  cerr << "VALUE: "<< value << endl;
  cerr << "SOLUTION:";
  for (size_t i = 0; i < x.size(); i++) cerr << " " << x[i];
  cerr << endl;
  return 0;
}
\end{lstlisting}


\subsection{Primes}

\begin{lstlisting}
// Primes less than 1000:
//      2     3     5     7    11    13    17    19    23    29    31    37
//     41    43    47    53    59    61    67    71    73    79    83    89
//     97   101   103   107   109   113   127   131   137   139   149   151
//    157   163   167   173   179   181   191   193   197   199   211   223
//    227   229   233   239   241   251   257   263   269   271   277   281
//    283   293   307   311   313   317   331   337   347   349   353   359
//    367   373   379   383   389   397   401   409   419   421   431   433
//    439   443   449   457   461   463   467   479   487   491   499   503
//    509   521   523   541   547   557   563   569   571   577   587   593
//    599   601   607   613   617   619   631   641   643   647   653   659
//    661   673   677   683   691   701   709   719   727   733   739   743
//    751   757   761   769   773   787   797   809   811   821   823   827
//    829   839   853   857   859   863   877   881   883   887   907   911
//    919   929   937   941   947   953   967   971   977   983   991   997
// Other primes:
//    The largest prime smaller than 10 is 7.
//    The largest prime smaller than 100 is 97.
//    The largest prime smaller than 1000 is 997.
//    The largest prime smaller than 10000 is 9973.
//    The largest prime smaller than 100000 is 99991.
//    The largest prime smaller than 1000000 is 999983.
//    The largest prime smaller than 10000000 is 9999991.
//    The largest prime smaller than 100000000 is 99999989.
//    The largest prime smaller than 1000000000 is 999999937.
//    The largest prime smaller than 10000000000 is 9999999967.
//    The largest prime smaller than 100000000000 is 99999999977.
//    The largest prime smaller than 1000000000000 is 999999999989.
//    The largest prime smaller than 10000000000000 is 9999999999971.
//    The largest prime smaller than 100000000000000 is 99999999999973.
//    The largest prime smaller than 1000000000000000 is 999999999999989.
//    The largest prime smaller than 10000000000000000 is 9999999999999937.
//    The largest prime smaller than 100000000000000000 is 99999999999999997.
//    The largest prime smaller than 1000000000000000000 is 999999999999999989.
\end{lstlisting}

\end{document}