\documentclass[10pt,letterpaper,twocolumn,twosided]{article}

\usepackage[utf8]{inputenc}
\usepackage[spanish]{babel}
\usepackage{listings}
\usepackage[usenames,dvipsnames]{color}
\usepackage{amsmath}
\usepackage{verbatim}
\usepackage{hyperref}
\usepackage{color}
\usepackage{geometry}

\geometry{verbose,landscape,letterpaper,tmargin=2cm,bmargin=2cm,lmargin=2cm,rmargin=2cm}
\newcommand{\codigofuente}[1]{
\verbatiminput{#1}
\dotfill
}

\setlength{\columnsep}{0.25in}
\setlength{\columnseprule}{1px}

\begin{document}

\title{Resumen de algoritmos para maratones de programación}
\author{Universidad Tecnológica de Pereira}
\maketitle

\tableofcontents
\lstloadlanguages{C++,Java}

%===============================================%
\section{Plantilla}

\codigofuente{../src/template.${EXT}}%$

%===============================================%
\section{Grafos}

\subsection{Depth First Search}

Es un algoritmo para recorrer o buscar en un grafo. Empieza visitando la raiz y luego explora 
a fondo cada rama antes de hacer el backtraing.\\
\\
Complejidad $O(|V|+|E|)$

\codigofuente{../src/graphs/dfs.${EXT}}%$

\subsection{Breadth First Search}

Es un algoritmo de búsqueda que empieza en la raíz y explora todos los nodos vecinos. Luego para cada
vecino repite el proceso hasta encontrar la meta.\\
Complejidad $O(|V|+|E|)$

\codigofuente{../src/graphs/bfs.${EXT}}%$

\subsection{Shortest path problem}

El shortest path problem es el problema de encontrar el camino mas corto entre dos nodos de un grafo. Existen
dos algoritmos para solucionarlo: Dijkstra y Bellman-Ford. Si bien Dijkstra tiene una complejidad mejor que
Bellman-Ford, no funciona para grafos con aristas con peso negativo mientras que Bellman-Ford si lo hace.

\subsubsection{Dijkstra}

Calcula la ruta más corta a todos los nodos desde un origen.\\
Todas las aristas deben ser NO negativas, en ese caso usar Bellman-ford.\\
Complejidad $O(E\ log V)$

\codigofuente{../src/graphs/dijkstra.${EXT}}%$


\subsubsection{Bellman-Ford}
Este algoritmo calcula la ruta más corta en un grafo dirigido, en el cual el peso de las aristas puede 
ser positivo o negativo. Para grafos con pesos NO-negativos, el algoritmo de Dijkstra resuelve más
rápido este problema.\\

%The Bellman–Ford algorithm computes single-source shortest paths in a weighted digraph. For graphs with
%only non-negative edge weights, the faster Dijkstra's algorithm also solves the problem. Thus, Bellman–Ford
%is used primarily for graphs with negative edge weights. The algorithm is named after its developers, Richard
%Bellman and Lester Ford, Jr.\\

Si un grafo contiene un ``Ciclo negativo'', por ejemplo, un ciclo cuya suma de aristas sea un valor negativo,
entonces camina de forma arbitraria con los pesos que puede ser construida, por ejemplo, no puede haber camino más corto.
El algoritmo puede detectar ciclos negativos y reportar su existencia, pero no puede producir una respuesta correcta
si un ciclo negativo es alcanzable desde el nodo origen.\\

%If a graph contains a "negative cycle", i.e., a cycle whose edges sum to a negative value, then walks of
%arbitrarily low weight can be constructed, i.e., there can be no shortest path. Bellman-Ford can detect
%negative cycles and report their existence, but it cannot produce a correct answer if a negative cycle is
%reachable from the source.\\
Para solucionar el longest path, simplemente se aplica bellman-ford con los pesos de los caminos negativos.
\begin{verbatim}

procedure BellmanFord(list vertices, list edges, vertex source)
   // This implementation takes in a graph, represented
   // as lists of vertices
   // and edges, and modifies the vertices so that 
   // their distance and
   // predecessor attributes store the shortest paths.
   // Step 1: initialize graph
   for each vertex v in vertices:
       if v is source then v.distance := 0
       else v.distance := infinity
       v.predecessor := null

   // Step 2: relax edges repeatedly
   for i from 1 to size(vertices)-1:
       for each edge uv in edges: // uv is the edge from u to v
           u := uv.source
           v := uv.destination
           if u.distance + uv.weight < v.distance:
               v.distance := u.distance + uv.weight
               v.predecessor := u

   // Step 3: check for negative-weight cycles
   for each edge uv in edges:
       u := uv.source
       v := uv.destination
       if u.distance + uv.weight < v.distance:
           error "Graph contains a negative-weight cycle"
\end{verbatim}

\subsection{Shortest pair of edge disjoint paths}

Edge disjoint shortest pair algorithm is an algorithm in computer network routing. The algorithm 
is used for generating the shortest pair of edge disjoint paths between a given pair of vertices as follows:\\

- Run the shortest pair algorithm for the given pair of vertices\\

- Replace each edge of the shortest path (equivalent to two oppositely directed arcs) by a single arc directed \\
towards the source vertex

- Make the length of each of the above arcs negative\\

- Run the shortest path algorithm (Note: the algorithm should accept negative costs)\\

- Erase the overlapping edges of the two paths found, and reverse the direction of the remaining arcs on the 
first shortest path such that each arc on it is directed towards the sink vertex now. The desired pair of 
paths results.\\


\subsection{All-pairs shortest paths}

Es una versión del shortest path problem donde hay que hayar la menor distancia entre todos los nodos.

Existen dos algoritmos para solucionarlo, Floyd-Warshall y Jhonson. Floyd-Warshall lo hace en tiempo cúbico, 
mientras que Jhonson en $V x E$, por lo que solo es útil si $E$ es asimptoticamente menor que $V x V$ (es decir,
el grafo no es muy denso).

\subsubsection{Floyd-Warshall}

In computer science, the Floyd–Warshall algorithm (sometimes known as the WFI Algorithm or Roy–Floyd algorithm)
is a graph analysis algorithm for finding shortest paths in a weighted graph (with positive or negative edge weights).
A single execution of the algorithm will find the lengths (summed weights) of the shortest paths between all pairs
of vertices. The algorithm is an example of dynamic programming.\\ 

Pseudocodigo:\\

Normal\\

/* 

    Assume a function edgeCost(i,j) which returns the cost of the edge from i to j
    (infinity if there is none).
    Also assume that n is the number of vertices and edgeCost(i,i) = 0

*/
 
 int path[][];

 /*

    A 2-dimensional matrix. At each step in the algorithm, path[i][j] is the shortest path
    from i to j using intermediate vertices $(1 .. k - 1)$.  Each path[i][j] is initialized to
    edgeCost(i,j) or infinity if there is no edge between i and j.

 */

 procedure FloydWarshall ()\\
    for k := 1 to n\\
       for i := 1 to n\\
          for j := 1 to n\\
             path[i][j] = min ( path[i][j], path[i][k]+path[k][j] );\\

Con reconstrucción de path

procedure FloydWarshallWithPathReconstruction ()\\
    for k := 1 to n\\
       for i := 1 to n\\
          for j := 1 to n\\
             if path[i][k] + path[k][j] < path[i][j] then\\
                path[i][j] := path[i][k]+path[k][j];\\
                next[i][j] := k;\\

 procedure GetPath (i,j)\\
    if path[i][j] equals infinity then\\
      return "no path";\\
    int intermediate := next[i][j];\\
    if intermediate equals 'null' then
      return " "; \\
   else\\
      return GetPath(i,intermediate) + intermediate + GetPath(intermediate,j);\\
      


\subsubsection{Johnson}

Johnson's algorithm is a way to find the shortest paths between all pairs of vertices in a
sparse directed graph. It allows some of the edge weights to be negative numbers, but no 
negative-weight cycles may exist. It works by using the Bellman–Ford algorithm to compute a
transformation of the input graph that removes all negative weights, allowing Dijkstra's algorithm 
to be used on the transformed graph.\\

Pseudocodigo:

Johnson's algorithm consists of the following steps:

First, a new node q is added to the graph, connected by zero-weight edges to each other node.\\

Second, the Bellman–Ford algorithm is used, starting from the new vertex q, to find for each vertex v the 
least weight h(v) of a path from q to v. If this step detects a negative cycle, the algorithm is terminated.\\

Next the edges of the original graph are reweighted using the values computed by the Bellman–Ford algorithm: 
an edge from u to v, having lengthw(u,v), is given the new length $ w(u,v) + h(u) - h(v).$\\

Finally, q is removed, and Dijkstra's algorithm is used to find the shortest paths from each node s to every
other vertex in the reweighted graph.\\

In the reweighted graph, all paths between a pair s and t of nodes have the same quantity $h(s) - h(t)$ added to
them, so a path that is shortest in the original graph remains shortest in the modified graph and vice versa.
However, due to the way the values h(v) were computed, all modified edge lengths are non-negative, ensuring the
optimality of the paths found by Dijkstra's algorithm. The distances in the original graph may be calculated from 
the distances calculated by Dijkstra's algorithm in the reweighted graph by reversing the reweighting transformation.\\

The time complexity of this algorithm, using Fibonacci heaps in the implementation of Dijkstra's algorithm, 
is O(V2log V + VE): the algorithm uses O(VE) time for the Bellman–Ford stage of the algorithm, and $O(V log V + E)$
for each of V instantiations of Dijkstra's algorithm. Thus, when the graph is sparse, the total time can be faster
than the Floyd–Warshall algorithm, which solves the same problem in time $ O(V3).$\\


\subsection{Shortest pair of completely disjoint paths}

In theoretical computer science and network routing, Suurballe's algorithm is an algorithm for finding two
disjoint paths in a nonnegatively-weighted directed graph, so that both paths connect the same pair of vertices
and have minimum total length. The algorithm was conceived by J. W. Suurballe and published in 1974. The main idea
of Surballe's algorithm is to use Dijkstra's algorithm to find one path, to modify the weights of the graph edges,
and then to run Dijkstra's algorithm a second time. The modification to the weights is similar to the weight 
modification in Johnson's algorithm, and preserves the non-negativity of the weights while allowing the second
instance of Dijkstra's algorithm to find the correct second path.

Suurballe's algorithm performs the following steps:

Find the shortest path tree T rooted at node s by running Dijkstra's algorithm. This tree contains for every vertex u,
a shortest path from s to u. Let P1 be the shortest cost path froms to t. The edges in T are called tree edges
and the remaining edges are called non tree edges.

Modify the cost of each edge in the graph by replacing the cost w(u,v) of every edge (u,v) 
by $w'(u,v) = w(u,v) - d(s,v) + d(s,u).$ According to the resulting modified cost function, all tree edges have a
cost of 0, and non tree edges have a non negative cost.

Create a residual graph Gt formed from G by removing the edges of G that are directed into s and by reversing 
the direction of the zero length edges along path P1.

Find the shortest path P2 in the residual graph Gt by running Dijkstra's algorithm.

Discard the reversed edges of P2 from both paths. The remaining edges of P1 and P2 form a subgraph with two 
outgoing edges at s, two incoming edges at t, and one incoming and one outgoing edge at each remaining vertex.
Therefore, this subgraph consists of two edge-disjoint paths from s to t and possibly some additional (zero-length)
cycles. Return the two disjoint paths from the subgraph.

\subsection{Minimum Spanning Tree}

\subsubsection{Algoritmo de kruskal}

\subsubsection{Algoritmo de Prim}

\subsection{Strongly Connected Components}

\subsection{Puntos de articulación}

\subsection{2-SAT}

\subsection{Maximum bipartite matching}

\subsection{Flujo Máximo}

\subsection{Lowest Common Ancestor: TarjanOLCA}

%===============================================%
\section{Matemáticas}

\subsection{Aritmética Modular}

Colección de códigos útiles para aritmética modular\\
\codigofuente{../src/euclidean.${EXT}}%$

\subsection{Mayor exponente de un primo que divide a n!}
\codigofuente{../src/mate/pow_div.${EXT}}%$


\subsection{Potencia modular}

\subsection{Criba de Eratóstenes}



%===============================================%
\section{Geometría}

\subsection{Utilidades Geometría}
Código base para implementación de otros algoritmos.
\codigofuente{../src/geom/utilities.${EXT}} %$

\subsection{Distancia mínima: Punto-Segmento}

\subsection{Distancia mínima: Punto-Recta}

\subsection{Determinar si un polígono es convexo}

\subsection{Determinar si un punto está dentro de un polígono convexo}

\subsection{Determinar si un punto está dentro de un polígono cualquiera}

\subsection{Intersección de dos rectas}

\subsection{Intersección de dos segmentos}

\subsection{Determinar si dos segmentos se intersectan o no}

\subsection{Centro del círculo que pasa por tres puntos}

\subsection{Par de puntos más cercanos}

\subsection{Par de puntos más alejados}

\subsection{Área de un polígono}

\subsection{Convexhull}

%===============================================%
\section{Strings}

\subsection{Knuth-Morris-Pratt KMP}

\subsection{Aho-Corasick}

\subsection{Suffix Array}

%===============================================%
\section{Teoría de Juegos}

%===============================================%
\section{Estructuras de Datos}

\subsection{Prefix Tree - Triee}

\subsection{Fenwick Tree}

\subsection{Interval Tree}

%===============================================%
\section{Hashing} %%http://eternallyconfuzzled.com/tuts/algorithms/jsw_tut_hashing.aspx

\subsection{FNV Hash}

\subsection{JSW Hash}

%===============================================%
\section{Miseláneo}

\subsection {Bitwise operations}
Operaciones útiles con bits.

\codigofuente{../src/bitwise.${EXT}}







\end{document}
