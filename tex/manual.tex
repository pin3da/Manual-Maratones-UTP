\documentclass[10pt,letterpaper,twocolumn,twosided]{article}

\usepackage[utf8]{inputenc}
\usepackage[spanish]{babel}
\usepackage{listings}
\usepackage[usenames,dvipsnames]{color}
\usepackage{amsmath}
\usepackage{verbatim}
\usepackage{hyperref}
\usepackage{color}
\usepackage{geometry}

\geometry{verbose,landscape,letterpaper,tmargin=2cm,bmargin=2cm,lmargin=2cm,rmargin=2cm}
\newcommand{\codigofuente}[1]{
\verbatiminput{#1}
\dotfill
}

\setlength{\columnsep}{0.25in}
\setlength{\columnseprule}{1px}

\begin{document}

\title{Resumen de algoritmos para maratones de programación}
\author{Diego Alejandro Martínez - Manuel Felipe Pineda}
\maketitle

\tableofcontents
\lstloadlanguages{C++,Java}


\section{Plantilla}
\codigofuente{../src/template.${EXT}}

\section{Grafos}

\subsection{Dijkstra}

\subsection{Bellman-Ford}

\subsection{Floyd-Warshall}

\subsection{Johnson}

\subsection{Minimum Spanning Tree: Kruskal}

\subsection{Minimum Spanning Tree: Prim}

\subsection{Breadth First Search}

\subsection{Depth First Search}

\subsection{Strongly Connected Components}

\subsection{Puntos de articulación}

\subsection{2-SAT}

\subsection{Maximum bipartite matching}

\subsection{Flujo Máximo}

\subsection{Lowest Common Ancestor: TarjanOLCA}

\section{Matemáticas}

\section{Geometría}

\section{Strings}

\section{Teoría de Juegos}

\section{Estructuras de Datos}

\section{Hashing} %%http://eternallyconfuzzled.com/tuts/algorithms/jsw_tut_hashing.aspx

\section{Miseláneo}







\end{document}
